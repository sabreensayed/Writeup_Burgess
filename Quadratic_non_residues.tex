\documentclass{report}
\usepackage[utf8]{inputenc}
\usepackage{amsmath}
\usepackage{amssymb}
\usepackage{amsthm}
\usepackage{mathabx}
\usepackage{nicefrac}
\usepackage{verbatim}
\usepackage{mathtools}
\newtheorem{lemma}{Lemma}
\newtheorem{theorem}{Theorem}

\newcommand{\ignore}[1]{}

\title{Deriving Burgess's Bound on the Distribution of Quadratic and Higher Non-Residues}
\author{
  Sabreen Syed\\
  \multicolumn{1}{p{.7\textwidth}}{\centering\emph{Department of Computer Science and Engineering\\
  Indian Institute of Technology Kanpur, India}}}
\date{February 2018}

\begin{document}

\maketitle

\chapter{Introduction}
%Aim: Make the reader aware about the applications of the topics covered, give brief summary of existing results, and equip reader with the preleminaries to read the rest of the paper.
%
\section*{Preleminaries}
%
%
\chapter{Qualitative Analysis}
%Aim: Introduce Weil's Theorem and then Lemma 2. Give a basic qualitative proof of the p^{1/2} bound and the Burgess bound for quadratic residues, and extend the result to a general residue. Generalise the S_h^{2r} result to the sum of qudratic residues of quadratic polynomial for an subset, explain if the p^{1/2} bound can be made in an interval and explain why further extensions cannot be made to p^{1/4}
%
\section{The global property of local sums}
%
%
%
%
\end{document}
