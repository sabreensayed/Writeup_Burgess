\documentclass{report}
\usepackage[utf8]{inputenc}
\usepackage{amsmath}
\usepackage{amssymb}
\usepackage{amsthm}
\usepackage{mathabx}
\usepackage{nicefrac}
\usepackage{verbatim}
\usepackage{mathtools}
\newtheorem{lemma}{Lemma}
\newtheorem{theorem}{Theorem}
\newtheorem*{claim*}{Claim}
\newenvironment{claimproof}[1]{\par\noindent\underline{Proof:}\space#1}{\hfill $\diamond$ \vspace{3mm} \par}

\newcommand{\ignore}[1]{}

\title{Deriving Burgess's Bound on the Distribution of Quadratic and Higher Non-Residues}
\author{
  Sabreen Syed\\
  \multicolumn{1}{p{.7\textwidth}}{\centering\emph{Department of Computer Science and Engineering\\
  Indian Institute of Technology Kanpur, India}}}
\date{February 2018}

\begin{document}

\maketitle

\chapter{Introduction}
%Aim: Make the reader aware about the applications of the topics covered, give brief summary of existing results, and equip reader with the preliminaries to read the rest of the paper.
%
\section{Symbols and conventions}
%Character is
%
\section{Preliminaries}
%Discuss general characters with first demonstarting exampleas with quadratic ones.
%
\section{Interval sums of quadratic residues}
%
%%%%%%%%%%%%%%%%%%%%%%%%%%%%%
\chapter{Qualitative Analysis}
%Aim: Introduce Weil's Theorem and then Lemma 2. Give a basic qualitative proof of the p^{1/2} bound and the Burgess bound for quadratic residues, and extend the result to a general residue. Generalise the S_h^{2r} result to the sum of quadratic residues of quadratic polynomial for a subset, explain if the p^{1/2} bound can be made in an interval and explain why further extensions cannot be made to p^{1/4+\delta}
 The starting point for our analysis is Lemma 1 derived from Weil's theorem, which is an important result exhibiting the low correlation between the distribution of values a square-free polynomial takes in $\mathbb{F}_p$, and the distribution of quadratic residues.
%
\begin{lemma}
\cite{burgess}Let $f(x)$ be a monic square-free polynomial, with coefficients from $\mathbb{Z}$, reducible into linear factors with degree $\nu$. Then,
$$\Big\lvert\sum\limits_{x=0}^{p-1}\chi_2(f(x))\Big\rvert\leq (\nu -1)p^{\nicefrac{1}{2}}$$
\end{lemma}
%
The sum of $\chi_2$ values in an interval determines the general density of quadratic residues and non-residues in it. The sum would cancel out to small magnitudes if the densities were nearly equal, or be closer in magnitude to the size of the interval if not. Thus, lemma 1 essentially tells us that a square free polynomial over $\mathbb{F}_p$ would evaluate to roughly equal number of quadratic residues and non-residues over $\mathbb{F}_p$. We deduce lemma 2 in the next section by constructing an appropriate polynomial exploiting lemma 1 for our purpose of finding low bounds on the number of continuous quadratic residues or continuous non-residues.
%
%
\section{A useful global sum and the $O(p^{\nicefrac{1}{2}})$ bound}
Let us consider the sum of $\chi_2$ values in an interval of size $h$.

$$S_h(x)=\sum\limits_{n=0}^{h-1}\chi_2(x+n)$$

Given Lemma 1, which sums $\chi_2$ values of a polynomial over the entire interval $0$ to $p-1$, we would like to construct a square-free polynomial which would help put a limit on the sum of $\chi_2$ values in any interval of size $h$. We do this by expanding the sum $\sum\limits_{x=0}^{p-1}(S_h(x))^{2r}$:

$$(S_h(x))^{2r}= \bigg(\sum\limits_{n=0}^{h-1}\chi_2(x+n)\bigg)^{\mathclap{2r}} =\sum\limits_{m_1=1}^h\sum\limits_{m_2=1}^h...\sum\limits_{\mathclap{m_{2r}=1}}^h\chi_2(x+m_1)\chi_2(x+m_2)...\chi_2(x+m_{2r})$$
$$\sum\limits_{x=0}^{p-1}(S_h(x))^{2r}=\sum\limits_{x=0}^{p-1}\sum\limits_{m_1=1}^h\sum\limits_{m_2=1}^h...\sum\limits_{m_{2r}=1}^h\chi_2(x+m_1)\chi_2(x+m_2)..\chi_2(x+m_{2r})$$
\begin{equation} \label{sumpoly}
\sum\limits_{x=0}^{p-1}(S_h(x))^{2r}=\sum\limits_{m_1=1}^h\sum\limits_{m_2=1}^h...\sum\limits_{m_{2r}=1}^h\sum\limits_{x=0}^{p-1}\chi_2\big((x+m_1)(x+m_2)..(x+m_{2r})\big)
\end{equation}

To use lemma 1, we need to separate the perfect square polynomials forming the sum from the rest. To achieve this, we separate the ordered sequences or `tuples' in (\ref{sumpoly}), $(m_1, m_2...m_{2r})$, into two sets based on the multiplicity of distinct values in the tuple:
\begin{itemize}
  \item Set 1: Those tuples which contain an even count of each number in them. This set forms a bijection with every perfect square polynomial contributing. Since there are $\frac{(2r)!}{r!}$ ways to choose pairs of positions in a tuple and there are $h^{r}$ ways of filling them with values from $1$ to $h$, so there are less than $(2r)^r h^{r}$ such polynomials. Since the maximum value contributed to the sum of $\chi_2$ values from each such polynomial is at most $1$, the contribution to the sum over all $p$ values of $x$ is at most $(2r)^r h^{r}p$.
  \item Set 2: The rest of the tuples, which similarly correspond to the leftover polynomials exactly. Since there are $h^{2r}$ polynomials in all, trivially at most $h^{2r}$ are not perfect square polynomials. These polynomials are of the form:
  $$(x+s_1)^{e_1}(x+s_2)^{o_2}...(x+s_k)^{e_k}$$
  Where $s_i, i\in 1...k$ are distinct members of the tuple corresponding to the polynomial and $e_i$ and $o_i$ are even or odd powers respectively of the $i^{th}$ term. Now, $\chi\big((x+s_1)^{e_1}(x+s_2)^{o_2}...(x+s_k)^{e_k}\big)$ evaluates the same as $\chi\big((x+s_2)^{o_2}...(x+s_{k'})^{o_{k'}}\big)$ (taking all odd-powered factors) for all except at most $r$ values of $x$ (corresponding to the even-powered factors evaluating to $0$). Thus, their contribution to the sum is not more than $h^{2r}(p^{\nicefrac{1}{2}}(2r-1)+r)$.
\end{itemize}
%
We state lemma 2, which directly follows from what we have calculated so far.
%
\begin{lemma}
Let $S_h(x)=\sum\limits_{n=1}^{h}\chi_2(x+n)$. Then,
\begin{equation} \label{lemma2}
\sum\limits_{x=0}^{p-1}(S_h(x))^{2r}\leq h^{r}(2r)^{r}p+h^{2r}r(2p^{\nicefrac{1}{2}}+1)
\end{equation}
\end{lemma}
%%
\subsection*{The $O(p^{\nicefrac{1}{2}})$ bound}
To prove the $O(p^{\nicefrac{1}{2}})$ bound on the longest interval of continuous quadratic residues or of quadratic non-residues, we show that if the $O(p^{\nicefrac{1}{2}})$ bound is not true, then LHS of (\ref{lemma2}) grows faster with $p$ than the RHS, creating a contradiction.\\
Let the longest interval of quadratic residues or of quadratic non-residues be given by $h$ for a given $p$. Thus, $\lvert S_h(z)\rvert=h$ for some $z\in \mathbb{F}_p$. Consider the LHS of (\ref{lemma2}) in that case. As $\lvert S_h(x+k)\rvert \geq \lvert S_h(x)\rvert-2k$,
\begin{equation} \label{intervcontr}
\sum\limits_{x=0}^{p-1}(S_h(x))^{2r}\geq\sum\limits_{x=z}^{z+\lfloor\nicefrac{h}{4}\rfloor}\lvert S_h(x)\rvert^{2r} \geq h^{2r}+(h-2)^{2r}+...+(h-2\lfloor\nicefrac{h}{4}\rfloor)^{2r} \geq \frac{h^{2r+1}}{2^{2r+2}}
\end{equation}
Thus, we can say from (\ref{lemma2}) and (\ref{intervcontr}):
$$ \frac{h^{2r+1}}{2^{2r+2}} \leq h^{r}(2r)^{r}p+h^{2r}r(2p^{\nicefrac{1}{2}}+1)$$
Now we set $r=1$ and divide both sides by $h^{2r}$. We get:
\begin{equation} \label{contreq}
\frac{h}{2^{4}} \leq 2h^{-1}p+(2p^{\nicefrac{1}{2}}+1)
\end{equation}
which has a disparity in the growth of the LHS and the RHS with respect to the inequality. This manifests itself for $h>33p^{\nicefrac{1}{2}}$: the LHS of (\ref{contreq}) then is greater than $(2+\frac{1}{16})p^{\nicefrac{1}{2}}$, but the RHS is bounded above by $(2+\frac{2}{33})p^{\nicefrac{1}{2}}+1$, which is less than the lower limit of the LHS for large enough $p$. Hence, $h\leq 33p^{\nicefrac{1}{2}}$ for large enough $p$ , and thus $h\in O(p^{\nicefrac{1}{2}})$.
%
%
\section{The amplification of an interval}
We saw in the previous section how we split an interval in $\mathbb{F}_p$ of size $O(p^{\nicefrac{1}{2}})$ into smaller sliding intervals and used lemma 2 to obtain the bound. What we do in this section is one step further: we show that an interval is equivalent to several small intervals in terms of the number of quadratic residues and non-residues in it. This would be useful to further amplify the contribution of an interval when applying lemma 2.

We construct small intervals by multiplying an interval $N,N+1,...N+H$ by $q^{-1}$, resulting in the elements $Nq^{-1}...(N+H)q^{-1}$ of $\mathbb{F}_p$. Take a general resulting element:
$$(N+k)q^{-1}=(N+nq+r)q^{-1}=Nq^{-1}+rq^{-1}+n$$
where $k=nq+r$ and $0\leq r<q$ in $\mathbb{Z}$

There are $q$ values of $Nq^{-1}+rq^{-1}$, one for each $r$. To each $Nq^{-1}+rq^{-1}$, we add consecutive values of $n$ starting from $0$ to $\nicefrac{\lfloor H-r\rfloor}{q}$ to obtain each resulting element of $\{Nq^{-1}...(N+H)q^{-1}\}$, which results in $q$ small intervals. These intervals are disjoint because distinct $(N+k)$ result in different values of $(N+k)q^{-1}$. Now, how do we arithmetically calculate the exact values of an interval $(N+r)q^{-1},(N+r)q^{-1}+1,...(N+r)q^{-1}+\lfloor\nicefrac{H-r}{q}\rfloor$? Since we are dealing with elements in $\mathbb{F}_p$, $N+r\equiv N+r+tp (mod\text{ }p), t\in \mathbb{Z}$. For exactly one $t\in\{0, 1, q-1\}$ for each $r$, $N+r+tp\divides q$, which makes $\nicefrac{(N+r+tp)}{q}$ an integer. Thus the intervals translate to integers between $\nicefrac{(N+r+tp)}{q}$ and $\nicefrac{(N+r+tp+H-r)}{q}$ for each value of $r$ with the corresponding value of $t$ which makes $N+r+tp\divides q$, i.e. integers between $\nicefrac{N+tp}{q}$ and $\nicefrac{(N+tp+H)}{q}$ for all $t\in\{0,1,...q-1\}$.

The small intervals constructed are equivalent due the fact that multiplying the interval by $q^{-1}$, preserves or swaps the original cardinality of quadratic residues and non-residues, since $\chi_2((N+k)q^{-1})=\chi_2(q^{-1})\chi_2(N+k)$, and $\chi_2(q^{-1})$ is a constant. We now prove an important theorem which solidifies the effect of the amplification.
%
\begin{lemma}
\cite{burgess}Given a set of $Q$ distinct natural numbers, whose general member is given by $q$, which are pairwise co-prime to each other, and any natural numbers $N$ and $H$ such that:
\begin{equation} \label{lemma3cond}
2Hq<p
\end{equation}
Let us denote by $I(q,t)$ the real interval
$$\frac{N+tp}{q} \leq z \leq \frac{N+H+tp}{q}$$
Then it is possible to associate with each $q$ a set $T(q)\subseteq \{0,1,\ldots q-1\}$ of size at least $q-Q$, such that $\forall q \forall t\in T(q)$, $I(q,t)$ are disjoint.
\end{lemma}
\begin{proof}
For a given $q$, say $q_o$, consecutive $I(q_o,t)$ differ by $\nicefrac{p}{q_o}$ in their starting points, and have a width of $\nicefrac{H}{q_o}$. Since $H<p$, $I(q_o,t)$ are disjoint. The $I(q_o,t)$ are in fact very narrow compared to the difference between their starting points, by a factor of less than $\nicefrac{1}{2q_o}$. We can thus consider the gaps between interval starting points as equal-width `blocks' of size $\nicefrac{p}{q_o}$, with the intervals as fringes of width $\nicefrac{H}{q_o}$. We make the following claim about two sets of blocks:
%
\begin{claim*}
  Let $r$ and $s$ be two co-prime numbers. From a point, say $y$, on the real line, lay blocks of size $r$ in the positive direction of the real axis to create points spaced by a gap of $r$. Similarly, from $y$, lay blocks of size $s$ in the positive direction. Then the first points which coincide are after $s$ blocks of size $r$ and $r$ blocks of size $s$.
\end{claim*}
\begin{claimproof}
  Since, $r$ and $s$ are co-prime, $gcd(r,s)=rs$. If, for the sake of contradiction, say $u<s$ blocks of size $r$ was the first coinciding point, then $ur\divides s\Rightarrow gcd(r,s) \leq ur < rs$, which is a contradiction.
\end{claimproof}
%
Now take two values of $q$, $q_1$ and $q_2$, and without loss of generality assume $q_1<q_2$,. The gap between intervals \ignore {clarify convention earlier} given by $I(q_1,t)$ for consecutive values of $t$ is $\nicefrac{p}{q_1}$, while the gap between intervals given by $I(q_2,t)$ for consecutive values of $t$ is $\nicefrac{p}{q_2}$. Let us scale the real line by multiplying it by $\nicefrac{q_1q_2}{p}$. The resulting gap blocks on this are then $q_2$ and $q_1$ respectively. Thus the point they first coincide are after $q_1$ blocks of $I(q_1,t)$ and $q_2$ blocks of $I(q_2,t)$. Thus in the unscaled real line, the unit blocks become of size $\nicefrac{p}{q_1q_2}$ and the first point which coincides is after a length of $p$. Since the overlapping region of $[\nicefrac{N}{q_1},\nicefrac{N+pq_1}{q_1})$ and $[\nicefrac{N}{q_2},\nicefrac{N+pq_2}{q_2})$ is not greater than $p$, there is only one point which coincides. With the fringes (actual intervals) of maximum width $\nicefrac{H}{q_1}$ lesser than the least possible seperation between intervals $\nicefrac{p}{q_1q_2}$, the fringes cannot intersect anywhere else as well.

Hence, for each $q$, there are at least $q-Q$ values of $t\in \{0,1,\ldots q-1\}$ such that $I(q,t)$ are disjoint from all other intervals associated with all $q$.

%The starting $I(q_o,t)$ are spread over an interval of size $p$, evenly spaced by $\nicefrac{p}{q_o}$
\begin{comment}
For a fixed $q$, $I(q,t)$ are disjoint as $H<p$. Now, for a pair of different values of $q$, say $q_1$ and $q_2$, let $I(q_1,t_1)$ and $I(q_2,t_2)$ intersect. Then we can say:
$$\frac{N+t_1p}{q_1} \leq \frac{N+H+t_2p}{q_2}$$
$$\text{and}\text{ }\frac{N+t_2p}{q_2} \leq \frac{N+H+t_1p}{q_1}$$
$$\Rightarrow N(q_2-q_1)+(t_1q_2-t_2q_1)p\leq Hq_2<\frac{1}{2}p$$
$$\Rightarrow N(q_1-q_2)+(t_2q_1-t_1q_2)p\leq Hq_1<\frac{1}{2}p$$
$$\Rightarrow N(q_2-q_1)-\frac{1}{2}p < (t_2q_1-t_1q_2)p <N(q_2-q_1)+\frac{1}{2}p$$
Since not more than a single multiple of $p$ can occur in a range of size less than $p$, $t_2q_1-t_1q_2$ can take only one value. Because $q_1$ and $q_2$ are co-prime and $0\leq t_1 \leq q_1$ and $0\leq t_2 \leq q_2$, there is only one pair of $t_1$ and $t_2$ values for a unique $t_2q_1-t_1q_2$. This is because for a unique valuation of $t_2q_1-t_1q_2$, the difference in the valuation of the first term must be same as that of the second term for different values of $t_2$. The values of $t_2q_1$ will be multiples of $q_1$, and for the difference in two such values to be the same as two values of $t_1q_2$, a multiple of $q_2$, the least possible difference can be the l.c.m $q_1q_2$, which cannot occur from given range of $t_1$ and $t_2$.

Hence, for each $q$, there are at least $q-Q$ values of $t\in \{0,1,\ldots q-1\}$ such that $I(q,t)$ are disjoint from all other intervals associated with all $q$.
\end{comment}
\end{proof}

We now proceed to prove the $O(p^{\nicefrac{1}{4}+\delta})$ bound on the least quadratic residue.
%
%
\section{The $p^{\nicefrac{1}{4}+\delta}$ bound}
We will prove the bound on the least quadratic non-residue by means of contradiction. Say $H>p^{\nicefrac{1}{4}+\delta}$ for some $\delta>0$, and $\Big \lvert \sum\limits_{x=N+1}^{N+H}\chi_p(x)\Big \rvert=H$. We know that for $H>cp^{\nicefrac{1}{2}}$, where $c$ is some constant, there can be no interval of all quadratic residues or all non-residues. Thus, we assume $H\leq cp^{\nicefrac{1}{2}}$\\
We take the pair-wise co-prime numbers $q$ required for the application of lemma 3 to be the primes in the range:
$$\frac{p^{\nicefrac{1}{4}}}{2} \leq q \leq p^{\nicefrac{1}{4}}$$
This satisfies the constraint in (\ref{lemma3cond}) for large enough $p$. The number of primes $q$ is given by $Q=\pi(p^{\nicefrac{1}{4}})-\pi(\frac{p^{\nicefrac{1}{4}}}{2})=o(p^{\nicefrac{1}{4}})$ by the prime number theorem. Now, restating the assumption:
$$\bigg \lvert\sum\limits_{x=N+1}^{N+H}\chi_p(x)\bigg \rvert=H$$
We proceed to transform this interval into several smaller intervals as discussed in the previous section. For a fixed $x$, $x+tp$ is divisible by $q$ for exactly one value of $t\in\{0,1,...q-1\}$. Also $\chi(x+tp)=\chi(x)$. Thus:
$$\bigg \lvert\sum\limits_{x=N+1}^{N+H}\chi_p(x)\bigg \rvert=\bigg \lvert\sum\limits_{t=0}^{t=q-1}\sum\limits_{\substack{x=N+1\\x+tp\divides q}}^{N+H}\chi_p(x+tp)\bigg \rvert=H$$
Since $x+tp\divides q$, thus $x+tp=qz$ for some $z\in\mathbb{Z}$, $\Rightarrow z=\frac{x+tp}{q}$.\\
For $N+1\leq x \leq N+H$, we see that all such $z$ are the integer values in $I(q,t)$ according to lemma 3 for the corresponding values of $t$ such that $x+tp\divides q$.
$$\bigg \lvert\sum\limits_{t=0}^{t=q-1}\sum\limits_{\substack{x=N+1\\x+tp\divides q}}^{N+H} \chi_p \Big(\frac{x+tp}{q}.q\Big)\bigg \rvert=H$$
$$\bigg \lvert\sum\limits_{t=0}^{t=q-1}\sum\limits_{z\in I(q,t)}\chi_p(z.q)\bigg \rvert=H$$
$$\bigg \lvert\sum\limits_{t=0}^{t=q-1}\sum\limits_{z\in I(q,t)}\chi_p(z)\chi_p(q)\bigg \rvert=H$$
$$\sum\limits_{t=0}^{t=q-1}\bigg \lvert\sum\limits_{z\in I(q,t)}\chi_p(z)\chi_p(q)\bigg \rvert \geq H$$
$$\sum_q\sum\limits_{t=0}^{t=q-1}\bigg \lvert\chi_p(q)\sum\limits_{z\in I(q,t)}\chi_p(z)\bigg \rvert \geq HQ$$
$$\sum_q\bigg \lvert\chi_p(q)\bigg\rvert\sum\limits_{t=0}^{t=q-1}\bigg\lvert\sum\limits_{z\in I(q,t)}\chi_p(z)\bigg \rvert \geq HQ$$
$$\sum_q\sum\limits_{t=0}^{t=q-1}\bigg \lvert\sum\limits_{z\in I(q,t)}\chi_p(z)\bigg \rvert \geq HQ$$
Seperating the possibly overlapping intervals from the disjoint intervals:
\begin{equation} \label{seperation}
\sum_q\sum\limits_{t\in T(q)}\bigg \lvert\sum\limits_{z\in I(q,t)}\chi_p(z)\bigg \rvert+\sum_q\sum\limits_{\substack{t=0\\t\not \in T(q)}}^{q-1}\bigg \lvert\sum\limits_{z\in I(q,t)}\chi_p(z)\bigg \rvert \geq HQ
\end{equation}
Now,
\begin{equation} \label{overlapping}
\sum_q\sum\limits_{\substack{t=0\\t\not \in T(q)}}^{q-1}\bigg\lvert\sum\limits_{z\in I(q,t) }\chi_p(z)\bigg \rvert \leq \sum\limits_q Q\bigg\lceil\frac{H}{q}\bigg\rceil \leq \sum\limits_q Q\frac{2H}{q} < 4Q^2 Hp^{\nicefrac{-1}{4}}
\end{equation}
since $\sum\limits_q q^{-1} < Q.\Big(\frac{p^{\nicefrac{1}{4}}}{2}\Big)^{-1}$. Using \ref{seperation} and \ref{overlapping},
$$\sum_q\sum\limits_{t\in T(q)}\bigg \lvert\sum\limits_{z\in I(q,t)}\chi_p(z)\bigg \rvert > HQ - 4Q^2 Hp^{\nicefrac{1}{4}}=HQ(1-4Qp^{\nicefrac{-1}{4}})>\frac{HQ}{2}$$
\begin{equation} \label{sliceineq}
\sum\limits_{m=0}^{h-1}\sum_q\sum\limits_{t\in T(q)}\bigg \lvert\sum\limits_{z\in I(q,t)}\chi_p(z)\bigg \rvert = h \sum_q\sum\limits_{t\in T(q)}\bigg \lvert\sum\limits_{z\in I(q,t)}\chi_p(z)\bigg \rvert > \frac{HhQ}{2}
\end{equation}
for large enough $p$, since $Q=o(p^{\nicefrac{1}{4}})$. Now we would like to use the contribution of the disjoint intervals in \ref{sliceineq} in applying lemma 2, by considering sliding window sums of size $h$. For this, we calculate an estimate of the sum obtained for one interval:
$$\sum\limits_{\mathclap {z\in I(q,t)}}\chi_p(z+m)\geq \sum\limits_{\mathclap{z\in I(q,t)}}\chi_p(z)-2m$$
$$\sum\limits_{m=0}^{h-1}\sum\limits_{z\in I(q,t)}\chi_p(z+m)\geq \sum\limits_{m=0}^{h-1}\sum\limits_{z\in I(q,t)}\chi_p(z)-2 \sum\limits_{m=0}^{h-1}m$$
$$\sum\limits_{z\in I(q,t)}\sum\limits_{m=0}^{h-1}\chi_p(z+m)\geq h\sum\limits_{\mathclap{z\in I(q,t)}}\chi_p(z)-h^2$$
$$\bigg \lvert\sum\limits_{z\in I(q,t)} \sum\limits_{m=0}^{h-1}\chi_p(z+m) \bigg \rvert \geq \bigg \lvert h\sum\limits_{\mathclap{z\in I(q,t)}}\chi_p(z)-h^2 \bigg \rvert$$
$$\sum\limits_{z\in I(q,t)}\bigg \lvert \sum\limits_{m=0}^{h-1}\chi_p(z+m) \bigg \rvert \geq h\bigg \lvert \sum\limits_{z\in I(q,t)}\chi_p(z)\bigg \rvert-h^2 $$
$$\sum_q\sum\limits_{t\in T(q)}\sum\limits_{z\in I(q,t)}\bigg \lvert \sum\limits_{m=0}^{h-1}\chi_p(z+m) \bigg \rvert \geq h\sum_q\sum\limits_{t\in T(q)}\bigg \lvert \sum\limits_{z\in I(q,t)}\chi_p(z)\bigg \rvert-Qp^{\nicefrac{1}{4}}h^2$$
since the number of the intervals given by lemma 3, $I(q,t)$, are less than $Qp^{\nicefrac{1}{4}}$. So from (\ref{sliceineq})
$$\sum_q\sum\limits_{t\in T(q)}\sum\limits_{z\in I(q,t)}\bigg \lvert \sum\limits_{m=0}^{h-1}\chi_p(z+m) \bigg \rvert > \frac{HhQ}{2}-Qp^{\nicefrac{1}{4}} h^2=Qh\big(\frac{H}{2}-hp^{\nicefrac{1}{4}}\big)$$
Taking $h<\frac{1}{4}p^{\nicefrac{-1}{4}}H$, and using the terminology from lemma 2,
$$\sum_q\sum\limits_{t\in T(q)}\sum\limits_{z\in I(q,t)}\lvert S_h(z) \rvert > \frac{HhQ}{4}$$
Using Holder's inequality,
$$\sum_q\sum\limits_{t\in T(q)}\sum\limits_{z\in I(q,t)}\lvert S_h(z) \rvert \leq \bigg ( \sum_q\sum\limits_{t\in T(q)}\sum\limits_{z\in I(q,t)}\lvert S_h(z) \rvert^{2r} \bigg )^{\nicefrac{1}{2r}}\bigg ( \sum_q\sum\limits_{t\in T(q)}\sum\limits_{z\in I(q,t)}1 \bigg )^{1-\nicefrac{1}{2r}}$$
$$\sum_q\sum\limits_{t\in T(q)}\sum\limits_{z\in I(q,t)}\lvert S_h(z) \rvert^{2r} \geq \bigg ( \sum_q\sum\limits_{t\in T(q)}\sum\limits_{z\in I(q,t)}\lvert S_h(z) \rvert \bigg )^{2r} \bigg ( \sum_q\sum\limits_{t\in T(q)}\sum\limits_{z\in I(q,t)}1 \bigg )^{-(2r-1)}$$
$$\sum_q\sum\limits_{t\in T(q)}\sum\limits_{z\in I(q,t)}\lvert S_h(z) \rvert^{2r} > \Big( \frac{HhQ}{4}\Big)^{2r}\Big ( Qp^{\nicefrac{1}{4}}.2.\frac{H}{p^{\nicefrac{1}{4}}}\Big )^{-(2r-1)}$$
$$\sum\limits_{z=0}^{p-1}\lvert S_h(z) \rvert^{2r}>\Big(\frac{1}{8}\Big)^{2r}HQh^{2r}$$
From lemma 2,
$$(2r)^{r}h^{r}p+h^{2r}r(2p^{\nicefrac{1}{2}}+1)>\Big(\frac{1}{8}\Big)^{2r}HQh^{2r}$$
For large enough $p$, the second term of the LHS diminishes compared to the RHS as $HQ\geq \frac{p^{\nicefrac{1}{2}+\delta}}{4\log p}$. Now, we take $h=\left \lfloor \frac{1}{5}p^{\nicefrac{-1}{4}}H \right \rfloor$. Since $h \geq \frac{1}{5}p^{\delta}$, we take $r>\delta^{-1}$, which results in the first term of the LHS diminishing with respect to the RHS.
Thus, we have arrived at a contradiction. Our assumption that $\Big \lvert \sum\limits_{x=N+1}^{N+H}\chi_p(x)\Big \rvert=H$ was wrong, hence the first quadratic non-residue is less than $p^{\nicefrac{1}{4}+\delta}$ for any $\delta>0$.
%
%
\section{Extension to general characters}
%Lemma 1 extension with reference to general characters
%
%Anlaogous proof of lemma 2 (but complete), without too deep an explanation, extend the o(p^1/2) simply
%
%Literally the same p^(1/4) section with extra modulus symbols. Very compact proof
Now we consider a natural extension of what we have proven so far: extending the result to general characters. Although a more general form of lemma 1 for general characters exists, a possible impediment in reapplying the former proof technique is dealing with the complex numbers involved in general characters. But as it turns out, we can easily extend the previously used techniques and theorems in the complex domain to obtain the same results for a general character.

The generalised form of lemma 1 for any character is:
\begin{lemma}
\cite{schmidt}Let $\chi$ be a non-trivial character of order $d$. Let $f(x)$ be a polynomial which is not a perfect $d^{th}$ power, with coefficients from $\mathbb{Z}$. Let $f(x)$ have $\nu$ distinct zeroes. Then,
$$\Big\lvert\sum\limits_{x=0}^{p-1}\chi(f(x))\Big\rvert\leq (\nu -1)p^{\nicefrac{1}{2}}$$
\end{lemma}
We henceforth assume $d \divides p-1$, because since $\chi^{d}(x)=1$ and $\chi(x^{p-1})=\chi^{p-1}(x)=1$ for $x=\neq0$, $\chi^{gcd(d,p-1)}(x)=1$ for $x=\neq0$, and hence any character of order $d$ is also a character of order $gcd(d,p-1)$, and hence we consider $gcd(d,p-1)$ as the character order.
We now use lemma 4 to derive an equivalent form of lemma 2. Again, we define
$$S_h(x)=\sum\limits_{n=0}^{h-1}\chi(x+n)$$
We then expand the sum $\lvert S_h(x)\rvert^{2r}$:
$$\lvert S_h(x)\rvert^{2r}= \Big\lvert\sum\limits_{n=0}^{h-1}\chi(x+n)\Big\rvert^{2r}= \Big(\sum\limits_{n=0}^{h-1}\chi(x+n)\Big)^r\Big(\overline{\sum\limits_{n=0}^{h-1}\chi(x+n)}\Big)^r$$
%\vspace*{-\baselineskip}
\begin{multline*}
\lvert S_h(x)\rvert^{2r} =\bigg(\sum\limits_{m_1=1}^h\sum\limits_{m_2=1}^h...\sum\limits_{\mathclap{m_{r}=1}}^h \chi(x+m_1)\chi(x+m_2)...\chi(x+m_{r})\bigg)
\\
\bigg(\sum\limits_{m_{r+1}=1}^h\sum\limits_{m_{r+2}=1}^h...\sum\limits_{m_{2r}=1}^h \overline{\chi(x+m_{r+1})\chi(x+m_{r+2})...\chi(x+m_{2r})}\bigg)
\end{multline*}
\begin{multline*}
\lvert S_h(x)\rvert^{2r} =\bigg(\sum\limits_{m_1=1}^h\sum\limits_{m_2=1}^h...\sum\limits_{\mathclap{m_{r}=1}}^h \chi\big((x+m_1)(x+m_2)...(x+m_{r})\big)\bigg)
\\
\bigg(\sum\limits_{m_{r+1}=1}^h\sum\limits_{m_{r+2}=1}^h...\sum\limits_{m_{2r}=1}^h \overline{\chi}\big((x+m_{r+1})(x+m_{r+2})...(x+m_{2r})\big)\bigg)
\end{multline*}
Using the property of a character that $\overline{\chi}(x)=\chi (x^{-1})$ for $x\neq0$:
\begin{multline*}
\lvert S_h(x)\rvert^{2r} =\bigg(\sum\limits_{m_1=1}^h\sum\limits_{m_2=1}^h...\sum\limits_{\mathclap{m_{r}=1}}^h \chi\big((x+m_1)(x+m_2)...(x+m_{r})\big)\bigg)
\\
\bigg(\sum\limits_{m_{r+1}=1}^h\sum\limits_{m_{r+2}=1}^h...\sum\limits_{m_{2r}=1}^h \chi\big((x+m_{r+1})^{p-2}(x+m_{r+2})^{p-2}...(x+m_{2r})^{p-2}\big)\bigg)
\end{multline*}
$$=\sum\limits_{m_1=1}^h\sum\limits_{m_2=1}^h...\sum\limits_{\mathclap{m_{2r}=1}}^h \chi\big((x+m_1)(x+m_2)...(x+m_{r})(x+m_{r+1})^{p-2}(x+m_{r+2})^{p-2}...(x+m_{2r})^{p-2}\big)$$
\begin{multline} \label{generaltuple}
\sum_{x=0}^{p-1}\lvert S_h(x)\rvert^{2r}= \sum\limits_{m_1=1}^h\sum\limits_{m_2=1}^h...\sum\limits_{\mathclap{m_{2r}=1}}^h
\\
\sum_{x=0}^{p-1}\chi\big((x+m_1)(x+m_2)...(x+m_{r})(x+m_{r+1})^{p-2}(x+m_{r+2})^{p-2}...(x+m_{2r})^{p-2}\big)
\end{multline}
In equation \ref{generaltuple}, we must seperate the perfect $d^{th}$ power polynomials from the rest. We do this, again as in the case of quadratic character, by examining the tuples formed by $(m_1, m_2...m_{2r})$. Now since $d \divides p-1\Rightarrow \gcd(d, p-2)=1$, no less than a power of $d$ of $(x+m)^{p-2}$ will form a perfect $d^{th}$ power. Thus a tuple represents a a perfect $d^{th}$ power polynomial if each element occuring in $m_1, m_2...m_{r}$ is paired with an occurence of the same element in $m_{r+1}, m_{r+2}...m_{2r}$, and the rest of the occurences of elements in $m_1, m_2...m_{r}$ are in a multiple of $d$, as well the rest of the occurences of elements in $m_{r+1}, m_{r+2}...m_{2r}$ are in a multiple of $d$. Thus the total number of such polynomials are:
$$\sum_{k=0}^{\lfloor\nicefrac{r}{d}\rfloor}\Bigg(\binom{r}{d}\binom{r-d}{d}...\binom{r-(k-1)d}{d}\Bigg)^2\binom{r-kd}{ 1}^2 \binom{r-kd+1}{1}^2...\binom{1}{1}^2$$
%
%
\section{Extensions to quadratic functions}
%
%
\section{Problems with attempts at further generalisation}
%2
%%%%%%%%%%%%%%%%%%%%%%%%%%%%%%%
\chapter{Quantitative Analysis}
%Aim:
\section{Energy}
%
%
\section{Sieving}
%
%
\section{}
\begin{thebibliography}{9}

\bibitem{burgess}
 D. A. Burgess,
\textit{The distribution of quadratic residues and non-residues},
Mathematika 4 (1957), 106-112.

\bibitem{burton}
David M. Burton, \textit{Elementary Number Theory},
McGraw Hill, 2012.

\bibitem{tonelli}
S. Lindhurst, \textit{An analysis of Shanks's algorithm for computing square roots in finite fields},
CRM Proceedings and Lecture Notes, Vol. 19 (1999) pp. 231–242.

\bibitem{polya}
G. Polya,
\textit{``{\"U}ber die Verteilung der quadratischen Eeste und Nichtreste''},
G{\"o}ttinger Nachrichten (1918), 21-29.

\bibitem{vinogradov}
I. M. Vinogradov,
\textit{``Sur la distribution des r{\'e}sidus et des non-r{\'e}sidus des puissances ''},
Journal Physico-Math. Soc. Univ. Perm, No. 1 (1918), 94-96.

\bibitem{schmidt}
W. Schmidt,
\textit{``Equations over finite fields, An elementary approach''},
Lecture Notes in Math. 536, Springer Verlag, 1976, p. 43

\end{thebibliography}
%
\end{document}
