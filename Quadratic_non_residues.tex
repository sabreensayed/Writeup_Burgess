\documentclass{report}
\usepackage[utf8]{inputenc}
\usepackage{amsmath}
\usepackage{amssymb}
\usepackage{amsthm}
\usepackage{mathabx}
\usepackage{nicefrac}
\usepackage{verbatim}
\usepackage{mathtools}
\newtheorem{lemma}{Lemma}
\newtheorem{theorem}{Theorem}

\newcommand{\ignore}[1]{}

\title{Deriving Burgess's Bound on the Distribution of Quadratic and Higher Non-Residues}
\author{
  Sabreen Syed\\
  \multicolumn{1}{p{.7\textwidth}}{\centering\emph{Department of Computer Science and Engineering\\
  Indian Institute of Technology Kanpur, India}}}
\date{February 2018}

\begin{document}

\maketitle

\chapter{Introduction}
%Aim: Make the reader aware about the applications of the topics covered, give brief summary of existing results, and equip reader with the preleminaries to read the rest of the paper.
%
\section{Symbols and conventions}
%Character is
%
\section{Preleminaries}
%
\section{Interval sums of quadratic residues}
%
%
\chapter{Qualitative Analysis}
%Aim: Introduce Weil's Theorem and then Lemma 2. Give a basic qualitative proof of the p^{1/2} bound and the Burgess bound for quadratic residues, and extend the result to a general residue. Generalise the S_h^{2r} result to the sum of qudratic residues of quadratic polynomial for a subset, explain if the p^{1/2} bound can be made in an interval and explain why further extensions cannot be made to p^{1/4+\delta}
In the previous chapter, we introduced Dirichlet characters given by $\chi$ and deduced some of their properties. One of their properties observed was that the global sum of a character is 0.  We now state Lemma 1 from \cite{burgess}, which is an important result exhibiting the low correlation between the distribution of values a square-free polynomial takes in $\mathbb{F}_p$, and the distribution of quadratic residues.
%
\begin{lemma}
Let $f(x)$ be a monic square-free polynomial, with coefficients from $\mathbb{Z}$, reducible into linear factors with degree $\nu$. Then,
$$\sum\limits_{x=0}^{p-1}\chi_p(f(x))\leq (\nu -1)p^{\nicefrac{1}{2}}$$
\end{lemma}
%
The sum of $\chi_p$ values in an interval determines the general density of quadratic residues and non-residues in it. The sum would cancel out to small magnitudes if the densities were nearly equal, or be closer in magnitude to the size of the interval if not. Thus, lemma 1 essentially tells us that a square free polynomial over $\mathbb{F}_p$ would evaluate to roughly equal number of quadratic residues and non-residues over $\mathbb{F}_p$. We deduce lemma 2 in the next section by constructing an appropriate polynomial exploiting lemma 1 for our purpose of finding low bounds on the number of continuous quadratic residues or continuous non-residues.
%
\section{A useful global sum}
%
\section{The $p^{\nicefrac{1}{2}}$ bound}
%
\section{The $p^{\nicefrac{1}{4}+\delta}$ bound}
%
\section{Extension to general characters}
%
\section{Extensions to quadratic functions}
%
\section{Problems with attempts at further generalisation}
%
%
\chapter{Quantitative Analysis}
%Aim:
\section{Energy}
%
%
\end{document}
